\chapter{攻读博士/硕士学位期间取得的研究成果} %博士/硕士记得选其一
\pubfont % 论文撰写规范里,这章是5号宋体,\pubfont 设置字号为5号了。但其实很多论文用小四号也OK。
% % =====================================
% % ============== 单盲版本 ==============
% % =====================================
一、已发表(包括已接受待发表)的论文,以及已投稿、或已成文打算投稿、或拟成文投稿的论文情况\underline{\textbf{(只填写与学位论文内容相关的部分):}}
\begin{table}
	\centering{}%
	\pubfont 
	\begin{longtable}{|>{\centering}m{0.5cm}|m{1.8cm}|>{\centering}m{2.8cm}|>{\centering}m{2.5cm}|>{\centering}m{2.2cm}|>{\centering}m{2.cm}|>{\centering}m{1cm}|}
		\hline 
		\textbf{序号} & \textbf{作者(全体作者,按顺序排列)} & \textbf{题 目} 						   & \textbf{发表或投稿刊物名称、级别} & \textbf{发表的卷期、年月、页码} & \textbf{与学位论文哪一部分(章、节)相关} &\textbf{被索引收录情况}\tabularnewline
		\hline 
		1    & 作者姓名、xxx、					  &  &  &  &  & \tabularnewline
		\hline 
		2	 & 	作者姓名、xxx、						&  			 &   &  &  & \tabularnewline
		\hline 
	\end{longtable}
\end{table}

注:在“发表的卷期、年月、页码”栏:

1.如果论文已发表,请填写发表的卷期、年月、页码;

2.如果论文已被接受,填写将要发表的卷期、年月;

3.以上都不是,请据实填写“已投稿”,“拟投稿”。

不够请另加页。

二、与学位内容相关的其它成果(包括专利、著作、获奖项目等)




% =============================================================
% =============================================================
% =============================================================



% % % =====================================
% % % ==============双盲版本 ==============
% % % =====================================
% 一、已发表(包括已接受待发表)的论文,以及已投稿、或已成文打算投稿、或拟成文投稿的论文情况\underline{\textbf{(只填写与学位论文内容相关的部分):}}
% % \begin{table}
% 	\centering{}%
% 	\pubfont 
%     \begin{longtable}{|>{\centering}m{5mm}|>{\centering}m{60mm}|>{\centering}m{19mm}|>{\centering}m{20mm}|>{\centering}m{19mm}|>{\centering}m{10mm}|} % 13.1
% 		\hline 
% 		\textbf{序号} & \textbf{发表或投稿刊物/会议名称} & \textbf{作者(仅注明第几作者)}   & \textbf{发表年份} & \textbf{与学位论文哪一部分(章、节)相关} &\textbf{被索引收录情况} \tabularnewline
% 		\hline 
        
% 		1    & IEEE Communications Surveys and Tutorials \\ SCI 1区期刊 & 第一作者  & 2020 & 第二章 &  SCI \tabularnewline
% 		\hline 
        
% 		2	 & Transactions of the Association for Computational Linguistics\\ JCR Q1 期刊 & 第一作者  & 2020 & 第三章 & SCI \tabularnewline
% 		\hline 

%         3	 & IEEE/CVF Computer Vision and Pattern Recognition Conference \\ CCF A 类会议 & 导师第一,本人第二 & 2022 & 第四章 & EI \tabularnewline
% 		\hline 
        
%         4	 & 2022 4th International Conference on Communications, Information System and Computer Engineering (CISCE) \\ EI 会议 & 导师第一,本人第二 & 2022 & 第五章 & EI \tabularnewline
% 		\hline

%         5	 &  xxx \\ EI 会议 & 导师第一,本人第二 & 20xx & 第x章 & EI \tabularnewline
% 		\hline
        
%         6	 &   &   &   &   &  \tabularnewline
% 		\hline
        
% 	\end{longtable}
% \end{table}

% 注:1.请在“作者”一栏填写本人是第几作者,例:“第一作者”或“导师第一,本人第二”等;

% 2.若文章未发表或未被接受,请在“发表年份”一栏据实填写“已投稿”,“拟投稿”。

% 不够请另加页。

% 二、与学位内容相关的其它成果(包括专利、著作、获奖项目等)

% 填写示例:

% 专利:已授权一项发明专利,第一发明人,2020

% 著作:参与编写一本著作,第二作者,2020

% 获奖项目:获省部级科技奖一项,第三获奖人,2019







%注:这部分一言难尽,我努力了很久都没有把这个表做好。感觉学校给的这个表的模板非常反人类。看国外大学的博士论文,那种像参考文献著录信息那样一行一行的,比较美观。而这个框框很难放文字进去。

\normalsize % \normalsize可以将下文调回和正文一样的字号,这个随个人喜好。注释掉的话,致谢就就跟随《攻读博士/硕士学位期间取得的研究成果》的字号。
